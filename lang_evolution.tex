In our LRM, we set out to create an extremely ambitious language.  Throughout
the course of our projet, we adjusted our goals, removing or modifying language
features as we saw fit.  The evolution of RAPID can be seen through the
examination of several features that existed in one form in our original LRM,
and exist in a different form in our current version of RAPID.

\subsection*{4.1.1 Errors}

In our LRM, we sought out to create a system of errors that included what we
called unsafe functions.  Unsafe functions would be labeled as unsafe, and then
returned their return type as well as an Error.  Operations like dictionary
accesses, list accesses, database operations, and the like would all be unsafe.

Here is an example unsafe function, from our LRM.

\begin{verbatim}
unsafe func access(dict<string, int> d, string key) int {
    int val, error error = d[key]
    return val, error
}
\end{verbatim}

Implementing Errors as desired was out of scope for us, and so we do not allow
unsafe functions.
The implications of this have been felt in other language features as well. We
wanted to allow users to return \texttt{null, Error} to throw an error, no
matter the return type.  To do this, we had to make all of our objects nullable.
Although we do not allow unsafe functions, we do allow multiple return types
for functions, as well as nullable primitives to faciltiate errros.

\subsection*{4.1.2 Database backing}

One of the features we hoped to implement in RAPID was the ability to persist
instances of classes in a postgres database.  Then, using API calls, users
could interact with their data over an API written in RAPID.  This code would
have been primarily written in Go however, and because our team was relatively
inexperienced in Go, what resources we could have spent implementing database
backing were instead spent on code generation.

\subsection*{4.1.3 Arguments to functions over HTTP}

Our LRM originally portrayed three different ways to pass data to a route: as
paramters within the URL path for the request, as query string arguments in the
request, and as JSON POSTed in the request body.  Because of this, we defined
our \texttt{http} functions to take in URL path params using \texttt{param}
blocks, query string arguments in the parenthesis of the function args, and
finally JSON as the final parameter.  The following defines code block from
our LRM describes our intention to do this.

\begin{verbatim}
http /* id */ ( /* path args */ /* formal args */ /* request body args */) {
    // statements
}
\end{verbatim}

However, because we were unable to facilitate arguments from the query string
or the response body, including the path args in the parenthesis of the function
is superfluous.  Despite this, that is how we have implemented http functions in
our current version of RAPID.

\subsection*{4.1.4 Summary}

In summary, our language has evolved and simplified greatly over the semester.
We are proud of RAPID, and it has been exciting to observe how the language has
changed over the course of its development.
