\subsection*{6.1 Representative Programs}


\subsubsection*{gcd\_server.rapid}

\begin{verbatim}

func gcd(int p, int q) int {
    while (q != 0) {
        int temp = q;
        q = p % q;
        p = temp;
    }
    return p;
}

namespace gcd {
    param int a {
        param int b {
            http (int a, int b) int {
                int res = gcd(a, b);
                return res;
            }
        }
    }
}

\end{verbatim}

\subsubsection*{gcd\_server.go}

\begin{verbatim}

package main

import (
    "fmt"
    "github.com/julienschmidt/httprouter"
    "log"
    "net/http"
)

var _ = fmt.Printf
var _ = http.StatusOK
var _ = log.Fatal
var _ = httprouter.CleanPath

func HTTPgcdab(w http.ResponseWriter, r *http.Request, XXX httprouter.Params) {
    var res Int

    tmp_8860581127662105769 := XXX.ByName("b")
    b := StringToInt(&tmp_8860581127662105769)
    _ = b
    tmp_8489902106938695875 := XXX.ByName("a")
    a := StringToInt(&tmp_8489902106938695875)
    _ = a
    tmp_3192161587922682755 := *a
    tmp_3651816005641979265 := *b
    res = gcd(&tmp_3192161587922682755, &tmp_3651816005641979265)

    tmp_6419075023523112259 := *res
    w.Write([]byte(*IntToString(&tmp_6419075023523112259)))
    return
}

func main() {

    router := httprouter.New()
    router.GET("/gcd/:a/:b/", HTTPgcdab)
    log.Fatal(http.ListenAndServe(":8080", router))

}
func gcd(p Int, q Int) Int {

    for {
        tmp_7985179176134664640 := *q
        tmp_4968362049263200281 := 0
        tmp_2152367932835595560 := *&tmp_7985179176134664640 != *&tmp_4968362049263200281
        if !(*(&tmp_2152367932835595560)) {
            break
        }
        var temp Int
        tmp_6575234515618809058 := *q
        temp = &tmp_6575234515618809058

        tmp_8175353689487733453 := *p
        tmp_7976305469002585692 := *q
        tmp_1330295997928933460 := *&tmp_8175353689487733453 % *&tmp_7976305469002585692
        q = &tmp_1330295997928933460

        tmp_8211254607639216761 := *temp
        p = &tmp_8211254607639216761
    }

    tmp_286320284431439979 := *p
    return &tmp_286320284431439979
}

\end{verbatim}

\subsubsection*{oop.rapid}

\begin{verbatim}

class User {
    int age;
    string name = "Stephen";
    optional int height;

    instance my {
        func is_old() boolean {
            return (my.age >= 30);
        }
        func make_older() {
           my.age = my.age + 1;
        }
    }
}

User stephen = new User(age=29);
println(stephen.age);
stephen.height = 73;
println(stephen.height);

if (stephen.is_old()) {
   println("Stephen is old");
}
else {
    println("Stephen is young");
}
stephen.make_older();
if (stephen.is_old()) {
   println("Stephen is old");
}

\end{verbatim}

\subsubsection*{oop.go}

\begin{verbatim}

package main

import (
	"fmt"
	"github.com/julienschmidt/httprouter"
	"log"
	"net/http"
)

var _ = fmt.Printf
var _ = http.StatusOK
var _ = log.Fatal
var _ = httprouter.CleanPath

type User struct {
	height Int
	name   String
	age    Int
}

func main() {
	tmp_7985179176134664640 := 29

	tmp_4968362049263200281 := "Stephen"
	stephen := User{
		age:    &tmp_7985179176134664640,
		height: nil,
		name:   &tmp_4968362049263200281,
	}

	_ = stephen
	tmp_6575234515618809058 := &stephen
	tmp_2152367932835595560 := (*tmp_6575234515618809058).age

	println(*tmp_2152367932835595560)
	tmp_8175353689487733453 := 73
	tmp_7976305469002585692 := &stephen
	(*tmp_7976305469002585692).height = &tmp_8175353689487733453

	tmp_8211254607639216761 := &stephen
	tmp_1330295997928933460 := (*tmp_8211254607639216761).height

	println(*tmp_1330295997928933460)

	tmp_3192161587922682755 := &stephen

	if *((*tmp_3192161587922682755).User__is_old()) {
		tmp_3651816005641979265 := "Stephen is old"
		println(*&tmp_3651816005641979265)
	} else {

		tmp_286320284431439979 := "Stephen is young"
		println(*&tmp_286320284431439979)
	}

	tmp_6419075023523112259 := &stephen

	(*tmp_6419075023523112259).User__make_older()

	tmp_8860581127662105769 := &stephen

	if *((*tmp_8860581127662105769).User__is_old()) {
		tmp_8489902106938695875 := "Stephen is old"
		println(*&tmp_8489902106938695875)
	}

}
func (my *User) User__make_older() {

	tmp_2861391897282324475 := &my
	tmp_5447851695057989743 := (*tmp_2861391897282324475).age

	tmp_5134325039177141690 := 1
	tmp_3910187507312597662 := *tmp_5447851695057989743 + *&tmp_5134325039177141690
	tmp_628303324353860165 := &my
	(*tmp_628303324353860165).age = &tmp_3910187507312597662

}

func (my *User) User__is_old() Bool {

	tmp_3602583590299280641 := &my
	tmp_3452337755216291150 := (*tmp_3602583590299280641).age

	tmp_5008436610057606955 := 30
	tmp_1025476952255935310 := *tmp_3452337755216291150 >= *&tmp_5008436610057606955
	return &tmp_1025476952255935310
}

\end{verbatim}

\subsection*{6.2 Test Suites}

The RAPID project utilized a complex test suite including multiple levels of integration testing. Tests in the \textbf{/parser} directory would only run the parser, tests in the \textbf{/sast} directory would invoke the parser and the semantic checker, and tests in the \textbf{/compiler} directory would do full compilation and output checking.

\subsection*{6.3 Why and How}

The test suite was used to ensure no regressions occured in the test suite and also as a means for guiding development. Most group members practiced Test Driven Development (TDD) in which they wrote failing test and then wrote code in the compiler to fix those failing tests.

The reason we had multiple levels of testing is so that members of the team could submit pull requests which would implement features only partially in the pipeline (implement only the parsing support, for example.)

\subsection*{6.4 Automation}

Our test suite ran for every pull request on our project using a tool called Travis CI, which would add the result of the test suite next to the pull request itself on the Github interface.
This insured that we would not inadvertently commit broken code.
In addition, the full test suite was run whenever the compiler was built using the \textbf{Make} command.

\subsection*{6.5 Testing Roles}
Each member of the team was responsible for writing and fully testing their own code through all the pipeline layers.
